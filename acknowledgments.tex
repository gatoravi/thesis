\thesisacknowledgments

First, I would like to thank my thesis mentor Barak Cohen. I was in a tough spot when I joined Barak's lab, a sinking ship I think Barak called it. The PI of the lab I had joined decided to move out of WashU towards the end of the second year of my PhD. I'm extremely thankful that Barak gave me an opportunity to join and train in his lab. Barak's mentoring is something I will really miss to say the very least. He put my development as a scientist at the forefront of my PhD, and was able to help me identify my strengths and weaknesses right at the start of my PhD (starting at the rotation.) He encouraged all of us to attend seminars, department events and always set an example by doing things himself first. He gave me complete independence as a computational student doing experiments to seek out external core/industry services that would make my experimental load lighter. He gave me complete freedom in my work, trusted me to make the right decisions and encouraged me to be fearless. The work in Chapter 3 just started as a conversation amongst three graduate students and Barak was supported and encouraged our ideas from the beginning. I could tell right from the beginning that the years I spent in Barak's lab were going to be some of the most formative years of my life.

Next, I would like to thank my wife Alicia for her kindness, constant fun presence and thoughtfulness from before and throughout graduate school. We got married and welcomed our son Arjun during graduate school. There were innumerable occasions of me complaining to her about something that didn't work and she helped me see my blind spots and the bigger picture on many such occasions. Her work in Global Health also helped me realize the massive present day challenges facing the world and helped me put my work in perspective. Thank you for being there, I truly couldn't have finished this thesis without you and I feel lucky to have you in my life. Thanks also for introducing me to your amazing parents, Karen and Juan, your family and many of your friends who later became my friends. They were all vested in my success during this program.

I would like to thank my Mom for always checking in on me during graduate school and for coming over to watch Arjun during the chaotic few months after his birth. I'd also like to thank my sister for watching over my mom in India. I'm grateful for the opportunity to live away from where I grew up and pursue this career, something that would have not been possible for me without the efforts of my parents.

I would like to thank my thesis commmittee members - Tim Schedl, Gary Stormo, Shankar Mukherji and Nate Stitziel. Every one of them met with me outside committee meetings and I took it upon myself to learn from their varied expertise. I'd especially like to thank my chair Tim Schedl. Tim helped me on numerous occasions through graduate school. He helped me figure out a lab when I had to find a new lab towards the end of my second year. There were innumerable occasions that I would just stop by Tim's office to chat about something interesting that I'd read or heard just to have a conversation and see what he thought about it. I found these conversations an amazing learning experience and also fun to just chat with him. I will miss these conversations. He was always welcoming and I feel really fortunate that I could receive his mentorship throughout graduate school. The Precision Medicine Pathway that Tim organized offered me innumerable occasions to interact with and learn from top researchers from around the world. Thank you Tim.

I'd also like to thank all the HSG faculty and the program directors John Rice and Nate Stitziel, and the previous director Pat for making the program such a friendly place to learn and do research. I'd like to especially thank Nancy Saccone for answering all my questions related to statistical genetics, hosting journal club and for helping work on a project related to Polygenic Risk Scores. I'd like to also thank my mentors prior to graduate school, Malachi Griffith and Obi Griffith for their support and mentorship when I worked in their lab. They were incredibly supportive of my decision to apply to graduate school while working in their lab. I'd like to also thank Don Conrad for sharing his insatiable passion for human genetics with me during the years I spent in his lab.

I would like to thank my lab mates for a fantastic intellectual atmosphere to do a PhD in and asking me the tough questions during lab meetings. I'll definitely miss our lab meeting environment. I'd like to especially acknowledge Siqi and Clarice, whom I worked closely with on the work described in Chapter 3. Attending both their weddings was one of the biggest highlights of my graduate school life. Siqi also helped me get started on experiments in the Cohen Lab and I owe a lot to him for transitioning from being completely lost to just occasionally lost. I'd also like to thank Dana King who was incredibly supportive when I first joined the lab. I always enjoyed chatting with Dustin Baldridge about human genetics research, sharing the same experimental 3T3 system with Dustin gave me some company in my solitary efforts. Also, thanks to everyone on the fourth floor of the Couch building for the many conversations over coffee/tea and lunch. I'll miss just sitting outside in the lobby area with a cup of tea and laptop in front of me.

WashU has been an incredible place for me to pursue my PhD, and I am deeply grateful for this opportunity. I joined WashU as a staff Bioinformaticist in August 2011, and my experiences here made me interested in applying to graduate school. My awe for the university grew with every passing year as I began to understand the incredible work that was being done here over such a long time. Walking through the halls where many scientific legends had walked and the pivotal human genome had been sequenced was not just intimidating but also inspiring. I would like to acknowledge DBBS for a fantastic training environment for my PhD. The program is so flexible in it's present form and let me pursue any research area that I was interested in. I'd especially like to acknowledge Sara Holmes for help behind the scenes with my PhD and all my fellow DBBS students I worked with at the Student Advisory Committee (SAC). Doing a PhD definitely took a toll on my physical and mental health but thanks to the fantastic health resources available to DBBS PhD students at Student Health this was manageable for me. 

I would like to personally thank Abul Usmani whom I know from my time in Don Conrad's lab for sharing his experimental knowledge with me and showing me around the wet-lab. Thanks also to Xuhua Chen from Rob Mitra's lab for sharing her wet-lab genius with me, the project in Chapter 2 and Chapter 3 would not be possible without her guidance with the 10X protocol. Chatting with Xuhua often helped me forget my wet-lab worries. Thanks to Jess, ML, Brian and Eric for enabling the CGS be a fantastic place to do genomics research.

I would like to acknowledge all my friends both in St. Louis and beyond - Harry, Jia, Ching, Kuan, Kalpit, Deepak, Niel. Thank you for directly or indirectly supporting me throughout this journey with your great company, often over boardgames, dinner, movies or cricket.

The work of Jon Kabat-Zinn \cite{ludwig2008j} helped me try to stay present, especially during stressful times and made me think deeply about the bigger picture of life. I'm  thankful that Jon switched to the work he does after starting out as a molecular biology scientist.

Finally, thanks to Sky and Arjun for sharing your playtime with me and reminding me of the joys of life everyday.

\null\hfill \thesisauthor

\noindent
\textit{Washington University in St.\@ Louis}\\
\textit{May 2023}
