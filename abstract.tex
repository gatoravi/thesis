\begin{abstract}

Gene expression is a stochastic process. A population of genetically identical cells grown in an identical environment can express different amounts of any given gene. This cell-to-cell variability in gene expression often underlies important phenotypic differences between cells. If we understand the sources of cell-to-cell variability in gene expression this can help uncover the mechanisms of transcription. The core of this thesis is split into two chapters related to cell-to-cell variability in gene expression. I first examine the sources and impact of cell-to-cell variability on a specific phenotype, a signaling pathway response. In the next chapter I discuss the results of using a novel technology that I co-developed to uncover the sources of cell-to-cell variability.

In the second chapter, I examine cell-to-cell variability in the Hedgehog signaling pathway response. The Hedgehog signaling pathway is an important developmental pathway. We examined cell-to-cell variability in the timing of genetically identical cells to respond to stimulation of this pathway using an established cell-culture model of hedgehog signaling. We identify two groups of cells in unstimulated cells, one that can respond very fast to hedgehog stimulation and another that responds on a slower timescale. We hypothesized that stochastic fluctuations in transcription factor activity underlie these differences between the two groups of cells. Using computational analysis of single-cell RNA sequencing data, at three different time-points after stimulation, I identified multiple transcription factors that are differentially expressed between the fast responders and the slow responders. Overexpression of four of these transcription factors can partly re-create the fast responder cell-state. I also found that overexpression of one of the transcription factors, \emph{Prrx1} is sufficient to drive the fast response. We conclude that stochastic cell-to-cell variability of \emph{Prrx1} underlies part of the cell-to-cell variability in how fast a given cell can respond to hedgehog pathway stimulation. An important follow-up question that I tackle in the next chapter is what are the factors that determine the cell-to-cell variability of different genes in the genome.

In the third chapter, I describe a novel technique I co-developed for measuring the factors underlying cell-to-cell variability across the genome. Specifically, I looked at the effect of genomic environments and cellular environments on cell-to-cell variability. Using this method, we integrate the same reporter in multiple genomic locations and measuring the expression of the reporter gene at a single-cell level. This helps us estimate the expression mean and variance of the same reporter gene at multiple genomic locations. Using this method, we also measure the global transcriptome of the same cells expressing the reporter. We then associated the cell-to-cell variability of the reporter at different genomic locations with the chromatin at the different genomic locations to uncover potential relationship between chromatin and cell-to-cell variability. We find that the chromatin marks that are associated with the cell-to-cell variability are often different from or have a different direction of association compared to the chromatin marks that are associated with the mean level of expression. We also used the global transcriptome to understand the effect of cell state on the cell-to-cell variability. Finally, using pairs of reporters observed in different cells we decomposed the total cell-to-cell variability into extrinsic and intrinsic noise and find that the cell-state influences the extrinsic noise. 

Cell-to-cell variability in gene expression, amongst genetically identical cells in the same environment can have functional consequences. One such consequence that I detected is the variability in the timing of response to hedgehog stimulation. The sources of cell-to-cell variability are important to understand both to further our fundamental understanding of the mechanisms of transcription and for practical reasons such as genome engineering and gene-therapy efforts. The genomic and cellular environment features associated with cell-to-cell variability uncovered in this thesis are good candidates for further examination of the sources of noise.

\end{abstract}