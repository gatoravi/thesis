\begin{abstract}

Gene expression is a stochastic process. A population of genetically identical cells grown in an identical environment can express different amounts of any given gene. This cell-to-cell variability in gene expression often underlies important phenotypic differences between cells. If we can understand the sources of cell-to-cell variability in gene expression this will further our understanding of the mechanisms of transcription. The core of this thesis is split into two chapters. One chapter I examine the impact of cell-to-cell variability on a specific phenotype, a signaling pathway response. In another chapter I discuss the results of using a novel technology that I developed to uncover the sources of cell-to-cell variability.

In the second chapter, I examine cell-to-cell variability in a signaling pathway response. The hedgehog signaling pathway is an important developmental pathway. We examine cell-to-cell variability in how fast genetically identical cells can respond to stimulation of this pathway using a well established cell-culture model of hedgehog signaling. We identify two groups of cells, one that can respond very fast to hedgehog stimulation and another that responds on a slower timescale. We hypothesized that stochastic  fluctuations in transcription factor activity could underlie functional differences between these subsets of cells. Using computational analysis of single-cell RNA sequencing data I identified multiple transcription factors that are differentially expressed between the fast responders and the slow responders. Overexpression of four of these transcription factors is sufficient to create the fast responder cell-state. I also found that overexpression of one of the transcription factors, \emph{Prrx1} is also sufficient to drive the fast response. We conclude that stochastic cell-to-cell variability of \emph{Prrx1} underlies part of the cell-to-cell variability in how fast a given cell responds to hedgehog pathway stimulation. An important follow-up question that I tackle in the next chapter is what are the factors that determine the cell-to-cell variability of different genes in the genome.

In the third chapter I helped develop a novel technique for measuring  the factors underlying cell-to-cell variability. Specifically, I looked at the effect of genomic environments and cellular environments on cell-to-cell variability. The method is called Single-cell Analysis of Reporter Gene Expression Noise and Transcriptome (SARGENT). Using this method we integrate the same reporter in multiple genomic locations and measuring the expression of the reporter gene at a single-cell level. This helps us estimate the expression mean and variance of the reporter gene at multiple genomic locations. Using SARGENT we also measure the global transcriptome of the same cells expressing the reporter. We then associated the cell-to-cell variability of the reporter at different genomic locations with the chromatin marks at the different genomic locations. We find that the chromatin marks that are associated with the cell-to-cell variability are often different from or have a different direction of association compared to the chromatin marks that are associated with the mean level of expression. We used the global transcriptome to understand the effect of cell state on the cell-to-cell variability. Finally, using pairs of reporters observed in different cells we decomposed the total cell-to-cell variability into extrinsic and intrinsic noise and find that the cell-state has an effect on the extrinsic noise. 

Cell-to-cell variability in gene expression, amongst genetically identical cells in the same environment, is often thought of as an inconvenience. However, this variabilitiy can have functional consequences like the variability in the timing of response to hedgehog stimulation. The sources of this variability are important to understand both to further our fundamental understanding of the mechanisms of transcription and for practical genome engineering and gene-therapy efforts. The features associated with cell-to-cell variability in this thesis are good candidates for further examination.

\end{abstract}