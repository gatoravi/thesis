\chapter{Discussion}
\label{chap:conclusion}
\tightlists

Recent years have shown that there is a lot of cell-to-cell variability in gene expression. However what does this variability mean? How does a cell respond to signals that are so variable between cells? How are biological processes regulated with high fidelity in the face of such variability. A lot of work has been carried out to answer these questions over the years. My thesis work provides a few more hints about how such systems operate.

In the second chapter of this thesis I looked at variability amongst cells responding to the hedgehog signaling pathway. I found that overexpression of a single transcription factor was able to make more cells respond faster. The cell-to-cell variability in gene expression of this single transcription factor could underlie important developmental decisions. It thus becomes important to understand what underlies the cell-to-cell variability in gene expressionof different genes.

In the second chapter of my thesis I helped develop a new technology to measure the cell-to-cell variability of the same reporter gene inserted in multiple genomic locations. By looking at how the cell-to-cell variability changes with the features of the different genomic locations I was able to understand what are the features that determine high cell-to-cell variability and low cell-to-cell variability. I also found that most of the cell-to-cell variability was determined by the mean level of expression, and removing the effect of the mean leaves a very small amount of variability that we studied.

In this chapter, I will dicuss the implications of these two pieces of work. Specifically, I will first touch on how I think the cell makes decisions in the face of so much cell-to-cell variability. Next, I will discuss whether we should study cell-to-cell variability in gene expression especially when most of the variability can be explained by the mean expression level. Finally, I will touch upon how this work can be expanded upon in the future.

\section{How are cellular decisions made in the face of noise?}

In the first project I showed that the cell-to-cell variability of a single transcription factor Prrx1 can influence whether a cell responds to a perturbation or not. One model that might explain how the cell responds is by using a threshold model for any cellular decision. The threshold model suggests that cells that have a level of protein above a certain level are the ones that are poised to respond. Cells that are below a protein level are the ones that don't respond. When a population of cells are induced to express a certain gene, the mean level of expression increases but often the mean level increase is accompanied by an increase in the cell-to-cell variability. However, the proportion of cells that respond to a stimuli is determined by the proportion of cells that are able to express the gene above a certain level. This model fits with what I observe in Prrx1.

In the model for hedgehog signaling in \label{Chap:hedgehog} we hypothesized that stochastic activity of transcription factors might underlie why some cells are able to respond faster to hedgehog perturbation vs others. We identified a lost of candidate transcription factors that are overexpressed in the fast responding cells and tested five TFs. We found that overexpressing one of the transcription factors Prrx1 is able to increase the proportion of cells that respond faster. In a threshold model of cellular decision making, cells that have sufficient activity of Prrx1, above a certain threshold, are able to respond to the hedgehog stimulation.  When we induce Prrx1 expression, we increase the mean and variance of Prrx1. However the proportion of cells that express Prrx1 above a certain threshold are able to respond. A follow-up question from this study is how many such regulators exist and how might we find them?

We identified the fast responder gene signature consisting of 300 candidates. In theory, there might be some false positives amongst this list of 300 genes. However, in our model a TF that is able to express a sufficient fraction of these 300 genes is able to make the cells respond. When we overexpress Prrx1, we showed that we misexpress 74 of the 300 genes. Could the 74 genes be the ones that matter? I think this question still remains to be answered. It could be that only a small fraction of the 74 genes matters. Or it could be the case that Prrx1 operates through a different mechanism where it turns on Gli2, and that's all that matters for the hedgehog signaling. We tried approaches looking for motifs in the 300 genes, and identifying enriched TFs in the promoter regions of the 300 genes, however we were limited by the number of TFs that we could test for their effect on the hedgehog response. 

My thesis consists of two main parts. 
In the first part I investigated cell-to-cell variability in the hedgehog pathway... We showed
In the second chapter we integrated the same reporter gene and looked at how the cell to cell variability changes across 
different integrations across the genome. .. We did.. we showed.. 

no stochastic activation of hedgehog pathway in 3T3 cells

Prrx1 can induce a faster response. how does it do it. how many other regulators are there.

\subsection{what do these results mean}


\subsection{what can we do by controlling noise}
- gene therapy

\subsection{Future Directions}
In this section I will describe future directions of the two projects that I worked on. I hope this section will provide somewhat useful to anyone studying cell-to-cell variability or graduate students in general.

In the second chapter, I was able to show how hedgehog signaling was faster in cells when I overexpressed a single transcription factor Prrx1. However, it remains unknown how many other transcription factors are able to do this. And is there a way to predict the genes that reduce this variability from the 300 candidates that we have computationally instead of testing them out one by one experimentally. We used approaches like cell-oracle \cite{cell-oracle} to predict which TF perturbation might change the cell state from the slow-responder to fast-respodner cell state. These approaches which integrate multiple lines of evidence including cell-state differential expression, transcription factor binding data, and chromatin accessibility data to predict the effects of TF perturbations. However, such approaches are still in their infancy and there is a lot of work that can be done here to develop computational approaches to predict perturbation effects. Data such as what I have generated for this thesis work - RNA-seq data after overexpressing multiple transcription factors either through plasmid transfection or in stably integrated cell-lines can be used to train such computational models. If these models are accurate enough, this would could potentially minimize the number of functional experiments and provide a list of high confidence genes to test.

On the technology side of things, in the second and third chapters we  used single-cell RNA sequencing to look at the cell-to-cell variability of different genes. Single-cell RNA sequencing, as discussed in Chapter 1, in-theory helps to measure the expression of all polyA transcripts within a cell. However, the current methods of transript capture are still far from perfect. Only a small fraction of the overall transcripts within a cell are captured \cite{pachter's postdoc} and it is unclear how random this dropout is amongst transcripts. I am optimistic that these technical challenges will be mostly solved over the next decade, this would further increase the statistical power of the approaches used in this thesis to detect true effects. The fold-changes that we detect in the expression of transcription factors between the slow responders and fast-responders is pretty small. More precise measurements of transcripts could potentially result in a smaller and more high-confidence set of candidate genes that are differentially expressed between sub-states. In Chapter 3, the mean independent noise is a small fraction of the overall noise levels of a gene. Having more precise measurements of noise will also improve our ability to Approaches such as merfish \cite{merfish} which use imaging to measure multiple transcripts within a cell could provide an alternative avenue to sequencing. However imaging approaches are currently very challenging to setup, and harder to scale compared to sequencing. This may very well change in the future.

We've used a computational method for tracking the trajectory that cells take during a cellular process, i.e hedgehog signaling. This approach is far from perfect, and in the future I expect there may be better technologies to trace the series of cell-state changes that a cell undergoes when confronted with a perturbation. Already barcode-based approaches are available \cite{morris lineaage tracing, arnav's paper}, however even these approaches are not fool proof and are largely expensive due to dropout issues. The availability of better lineage tracing technologies will further the study of cell-to-cell variability and shed light on why approaches like reprogramming are inefficient.

In chapter 2, we've integrated the same reporter gene in multiple locations to understand what features are associated with cell-to-cell variability. In the future, ony might imagine integrating different reporter genes or the same gene with different cis-regulatory elements genome-wide and study cell-to-cell variability in a similar approach. Indeed similar approaches have been used to study the impact of genomic location on mean expression \cite{claraice's papers}, however precise single-cell measurements will extend this to studying cell-to-cell variability. This approach would be invaluable for gene-therapy approaches, if there are interactions between a gene and it's local environment, and enable choosing locations with low variability. The same cis-regulatory element might have the same mean at different locations but different cell-to-cell variability, current bulk approaches that study just the mean level of expression are blind to such interesting phenomena.

hedgehog:
molecular barcoding instead of trajectory analysis
increase accessible cholesterol instead of srebf2
How can we identify other TFs that drive noise?



cas:
scaling up to more number of cells using scirnaseq


GAps that i filled:
- I used single-cell sequencin and trajectory analyssis to identify variability
- single-cell RNA-seq to look at chromatin effect, could b edone more
genomic location effect on mRNA noise
